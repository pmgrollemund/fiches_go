\documentclass[preview, border=0pt, varwidth=false]{standalone}

\usepackage[paperwidth=10.2cm, paperheight=14.65cm, margin=0cm]{geometry}
\usepackage{psgo}
\usepackage{unicode-math}
\setmathfont{XITSMath-Regular}

\begin{document}
	\setgounit{0.6cm} 
	
\parbox[c][14.65cm][c]{10.2cm}{
	\centering
	

	\begin{psgopartialboard}{(1,1)(11,9)}
		\stone{white}{c}{4}
		\stone{black}{e}{3}
		\stone{white}{j}{3}
		\pass
		\pass*
		\move{d}{5}
		\move{c}{5}
		\move{d}{6}
		\move{c}{7}
		\move{c}{3}
		\move{d}{7}
		\move{d}{4}
		\move{f}{7}
		\move{c}{6}
		\move{b}{6}
		\move{b}{3}
	\end{psgopartialboard}
	
	\vspace{1em}
	Les deux pierres blanches peuvent être laissées à la prise afin de profiter de l'initiative. Noir et blanc en ressortent également satisfaits. 
	}

\end{document}