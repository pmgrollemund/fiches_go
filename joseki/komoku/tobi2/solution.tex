\documentclass[preview, border=0pt, varwidth=false]{standalone}

\usepackage[paperwidth=10.2cm, paperheight=14.65cm, margin=0cm]{geometry}
\usepackage{psgo}

\begin{document}
	\setgounit{0.4cm} 
	
\parbox[c][14.65cm][c]{10.2cm}{
	\centering
	

	\begin{psgopartialboard}{(1,1)(7,11)}
		\stone{black}{d}{3}
		\stone{white}{d}{5}
		\pass*
		\pass
		\move{c}{5}
		\move{c}{6}
		\move{c}{4}
		\move{d}{6}
		\move{f}{3}
		\move{d}{10}
		\markpos{\marktr}{d}{7}
	\end{psgopartialboard}
	
	\vspace{1em}
	Blanc joue la connexion \stone[4]{white} afin de pouvoir profiter d'une belle extension sur le bord, sans se créer de faiblesse, comme aurait pu le faire un connexion diagonale en $\triangle$ à la place de \stone[4]{white}.

}

\end{document}