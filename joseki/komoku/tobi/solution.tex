\documentclass[preview, border=0pt, varwidth=false]{standalone}

\usepackage[paperwidth=10.2cm, paperheight=14.65cm, margin=0cm]{geometry}
\usepackage{psgo}
\usepackage{amsmath}

\begin{document}
	\setgounit{0.6cm} 
	
\parbox[c][14.65cm][c]{10.2cm}{
	\centering
	
	\begin{psgopartialboard}{(1,1)(11,9)}
		\stone{black}{c}{4}
		\stone[\marklb{A}]{black}{j}{3}
		\pass
		\move{e}{4}
		\move{d}{3}
		\move{d}{5}
		\move{c}{5}
		\move{d}{7}
		\move{c}{6}
		\move{g}{7}
		\markpos{\marktr}{e}{3}
	\end{psgopartialboard}
	
	\vspace{1em}
	
	La bonne réponse est \stone[2]{black} pour protéger le coin. Blanc ne pourrait pas descendre en $\triangle$ car il n'aurait pas l'espace suffisant pour son extension à cause de \stone[\marklb{A}]{black}.
	Il doit ensuite faire \stone[3]{white} afin de s'offrir de l'espace en jouant léger. En réponse à \stone[6]{black}, il protège la menace de couper le tobi par \stone[7]{white}, tout en continuant de se donner de l'espace.
	Noir profite de l'initiave et laisse blanc flotter au dessus de la $4^\text{ème}$ ligne.
}

\end{document}