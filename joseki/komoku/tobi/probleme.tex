\documentclass[preview, border=0pt, varwidth=false]{standalone}

\usepackage[paperwidth=10.2cm, paperheight=14.65cm, margin=0cm]{geometry}
\usepackage{psgo}
\usepackage{unicode-math}
\setmathfont{XITS Math}

\begin{document}
	\setgounit{0.6cm} 
	
\parbox[c][14.65cm][c]{10.2cm}{
	\centering
	
	{\Large\textbf{Approche en keima} 	$\medblackstar \medblackstar \medwhitestar \medwhitestar$ \\ Noir laisse le coin et blanc crée des faiblesses}
	\vspace{3em}
	
	\begin{psgopartialboard}{(1,1)(11,7)}
		\stone{black}{d}{4}
		\pass
		\move{f}{3}
		\move{h}{3}
	\end{psgopartialboard}
	
	\vspace{1em}
	Blanc approche et se fait pincer par \stone[2]{black}. \\ Une réponse moderne consiste à envahir le coin et pousser \\ sur les bords du groupe pour créer des faiblesses.
}

\end{document}