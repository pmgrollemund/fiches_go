\documentclass[preview, border=0pt, varwidth=false]{standalone}

\usepackage[paperwidth=10.2cm, paperheight=14.65cm, margin=0cm]{geometry}
\usepackage{psgo}
\usepackage{amsmath}

\begin{document}
	\setgounit{0.6cm} 
	
\parbox[c][14.65cm][c]{10.2cm}{
	\centering
	
	\begin{psgopartialboard}{(1,1)(11,11)}
		\stone{black}{c}{4}
		\stone[\marktr]{black}{j}{3}
		\stone{white}{d}{4}
		\stone[\marksq]{black}{d}{10}
		\markpos{\marklb{B}}{d}{9}
		\markpos{\marklb{A}}{c}{9}
		\pass*
		\pass
		\move{d}{3}		
		\move{c}{5}		
		\move{e}{4}		
		\move{d}{5}		
		\move{c}{3}		
		\move{e}{5}		
		\move{f}{4}		
		\move{c}{8}		
		\move{b}{9}		
		\move{f}{5}		
		\move{g}{5}		
		\move{g}{6}		
		\move{h}{6}		
		\move{g}{7}		
	\end{psgopartialboard}
	
	\vspace{1em}	
Noir joue atari avec \stone[3]{black} afin de poursuivre en direction de \stone[\marktr]{black}. 

\bigskip

Avant de monter en \stone[10]{white}, blanc consolide sa base de vie avec \stone[8]{white}. Etant contraint par \stone[\marksq]{black}, blanc ne peut pas faire son extension naturelle en A ou B, et se contente de \stone[8]{white}. En réponse, noir joue sur la $2^\text{ème}$ afin de couper la base de vie potentielle de blanc sur le bord.


\bigskip

Au terme de la séquence, noir profite de l'initiative pour jouer ailleurs, mais souffre de la présence de \stone[\marktr]{black} qui est surnuméraire dans cette configuration.}

\end{document}