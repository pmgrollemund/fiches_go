\documentclass[preview, border=0pt, varwidth=false]{standalone}

\usepackage[paperwidth=10.2cm, paperheight=14.65cm, margin=0cm]{geometry}
\usepackage{psgo}
\usepackage{unicode-math}
\setmathfont{XITSMath-Regular}

\begin{document}
	\setgounit{0.4cm} 
	
\parbox[c][14.65cm][c]{10.2cm}{
	\centering
	

	\begin{psgopartialboard}{(1,1)(11,9)}
		\stone{white}{c}{4}
		\stone{black}{e}{3}
		\stone{white}{j}{3}
		\pass
		\pass*
		\move{d}{5}
		\move{c}{5}
		\move{d}{6}
		\move{c}{7}
		\move{c}{3}
		\move{d}{4}
		\move{e}{4}
		\move{d}{3}
		\move{c}{6}
		\move{b}{6}
		\move{b}{7}
	\end{psgopartialboard}
	
	\vspace{1em}
	Noir gagne la possibilité de jouer \stone[11]{black} qui est un bon coup, parce qu'il est protégé de la capture sous couvert de capturer les 4 pierres blanches : 



	\begin{psgopartialboard}{(1,1)(11,9)}
		\stone{white}{c}{4}
		\stone{black}{e}{3}
		\stone{white}{j}{3}
		\pass
		\pass*
		\stone{black}{d}{5}
		\stone{white}{c}{5}
		\stone{black}{d}{6}
		\stone{white}{c}{7}
		\stone{black}{c}{3}
		\stone{white}{d}{4}
		\stone{black}{e}{4}
		\stone{white}{d}{3}
		\stone{black}{c}{6}
		\stone{white}{b}{6}
		\stone[\markma]{black}{b}{7}
		\pass
		\stone[\small$\mathsf{1}$]{white}{b}{8}
		\stone[\small$\mathsf{2}$]{black}{b}{5}
		\stone[\small$\mathsf{3}$]{white}{a}{7}
		\stone[\small$\mathsf{4}$]{black}{b}{4}
		\stone[\small$\mathsf{5}$]{white}{d}{2}
		\stone[\small$\mathsf{6}$]{black}{c}{2}
	\end{psgopartialboard}
	}

\end{document}