\documentclass[preview, border=0pt, varwidth=false]{standalone}

\usepackage[paperwidth=10.2cm, paperheight=14.65cm, margin=0cm]{geometry}
\usepackage{psgo}
\usepackage{unicode-math}
\setmathfont{XITSMath-Regular}

\begin{document}
	\setgounit{0.6cm} 
	
\parbox[c][14.65cm][c]{10.2cm}{
	\centering
	
	\begin{psgopartialboard}{(1,1)(11,7)}
		\stone{black}{c}{4}
		\stone{black}{j}{3}
		\markpos{\marktr}{d}{5}
		\pass
		\move{d}{4}
		\move{c}{5}
		\move{d}{3}
		\move{c}{3}
		\move{c}{2}
		\move{b}{2}
		\move{e}{2}
		\move{c}{1}
		\move{e}{6}
		\move{f}{3}
		\move{h}{6}
	\end{psgopartialboard}
	
	\vspace{1em}
	Blanc crée une forme légère jusqu'à \stone[9]{white}, et si noir attaque la forme blanche avec \stone[10]{black}, blanc répond légèrement avec \stone[11]{white} en étant prêt à sacrifier des pierres si noir coupe avec $\triangle$.
	}

\end{document}