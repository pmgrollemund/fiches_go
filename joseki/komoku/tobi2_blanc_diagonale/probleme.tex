\documentclass[preview, border=0pt, varwidth=false]{standalone}

\usepackage[paperwidth=10.2cm, paperheight=14.65cm, margin=0cm]{geometry}
\usepackage{psgo}
\usepackage{unicode-math}
\setmathfont{XITSMath-Regular}

\begin{document}
	\setgounit{0.4cm} 
	
\parbox[c][14.65cm][c]{10.2cm}{
	\centering
	
	{\Large\textbf{Approche en tobi} $\medblackstar \medblackstar \medblackstar \medwhitestar$ \\ Connexion diagonale de blanc}
	\vspace{3em}
	
	\begin{psgopartialboard}{(1,1)(9,9)}
		\stone{black}{d}{3}
		\pass
		\move{d}{5}
		\move{c}{5}
		\move{c}{6}
		\move{c}{4}
		\move{d}{7}
		\pass
		\move{e}{3}
	\end{psgopartialboard}
	
	\vspace{1em}
	La connexion en diagonale \stone[5]{white} apporte ici une faiblesse. Au regard et au profit de cette faiblesse, noir peut tenuki. Puis, si blanc compresse en \stone[7]{white}, noir consolide son coin et fragilise blanc. 
}

\end{document}