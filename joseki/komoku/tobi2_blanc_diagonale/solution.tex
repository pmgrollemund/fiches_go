\documentclass[preview, border=0pt, varwidth=false]{standalone}

\usepackage[paperwidth=10.2cm, paperheight=14.65cm, margin=0cm]{geometry}
\usepackage{psgo}

\begin{document}
	\setgounit{0.4cm} 
	
\parbox[c][14.65cm][c]{10.2cm}{
	\centering
	

	\begin{psgopartialboard}{(1,1)(9,9)}
		\stone{black}{d}{3}
		\stone{white}{d}{5}
		\stone{black}{c}{5}
		\stone{white}{c}{6}
		\stone{black}{c}{4}
		\stone[\marksq]{white}{d}{7}
		\pass
		\stone{white}{e}{3}
		\pass*
		\move{d}{4}
		\move{e}{4}
		\move{e}{5}
		\move{d}{6}
		\move{e}{2}
		\move{f}{2}
		\move{d}{2}
		\move{g}{4}
		\markpos{\marktr}{f}{5}
	\end{psgopartialboard}
	
	
	\vspace{1em}
	Après \stone[8]{white} noir a de nouveau l'initiative pour jouer ailleurs, ce qui correspond à un deuxième coup gratuit pour noir dans la globalité de cette séquence. De plus, à tout moment, noir peut jouer en $\triangle$ et mettre en difficulté blanc.

	\bigskip
	
	Ceci résulte de la connexion trop faible \stone[\marksq]{white} au début de la séquence.

}

\end{document}