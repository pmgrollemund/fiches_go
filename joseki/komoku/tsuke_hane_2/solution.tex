\documentclass[preview, border=0pt, varwidth=false]{standalone}

\usepackage[paperwidth=10.2cm, paperheight=14.65cm, margin=0cm]{geometry}
\usepackage{psgo}
\usepackage{amsmath}

\begin{document}
	\setgounit{0.6cm} 
	
\parbox[c][14.65cm][c]{10.2cm}{
	\centering
	
	\begin{psgopartialboard}{(1,1)(11,11)}
		\stone{black}{c}{4}
		\stone{black}{j}{3}
		\stone{white}{d}{4}
		\stone{black}{d}{10}
		\pass*
		\pass
		\move{d}{5}		
		\move{c}{3}		
		\move{c}{5}		
		\move{d}{3}		
	\end{psgopartialboard}
	
	\vspace{1em}	
Blanc profite que noir laisse le coin et se stabilise rapidement. 

\bigskip

Noir ne peut pas profiter d'une attaque critique et rapide sur ce groupe, donc noir a l'initiave mais reste insatisfait d'avoir perdu le coin.}

\end{document}