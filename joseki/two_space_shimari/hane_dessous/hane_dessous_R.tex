\documentclass[preview, border=0pt, varwidth=false]{standalone}

\usepackage[paperwidth=10.2cm, paperheight=14.65cm, margin=0cm]{geometry}
\usepackage{psgo}

\begin{document}
	\setgounit{0.6cm} 
	
	\parbox[c][14.65cm][c]{10.2cm}{
		\centering
		
		\begin{psgopartialboard}{(1,1)(9,7)}
			\stone{black}{c}{4}
			\stone{black}{f}{4}
			\stone{white}{h}{3}
			\pass
			\move{d}{4}
			\move{d}{3}
			\move{e}{3}
			\move{e}{4}
			\move{d}{5}
			\move{c}{3}
			\move{f}{3}
			\move{g}{4}
			\move{h}{4}
			\move{f}{6}
			\move{h}{6}
			\move{d}{6}
		\end{psgopartialboard}
		
		\vspace{1em}
		Noir capture les deux pierres blanches, blanc développe son influence sur le bord droit. Une fragilité demeure en G3.
	}
	
\end{document}