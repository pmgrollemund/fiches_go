\documentclass[preview, border=0pt, varwidth=false]{standalone}

\usepackage[paperwidth=10.2cm, paperheight=14.65cm, margin=0cm]{geometry}
\usepackage{psgo}

\begin{document}
	\setgounit{0.5cm} 
	
	\parbox[c][14.65cm][c]{10.2cm}{
		\centering
		
		\begin{psgopartialboard}{(1,1)(19,8)}
			\stone{black}{c}{4}
			\stone{black}{f}{4}
			\stone{white}{h}{3}
			\stone{white}{q}{4}
			\pass
			\move{d}{4}
			\move{d}{3}
			\move{e}{3}
			\move{e}{4}
			\move{d}{5}
			\move{c}{3}
			\move{f}{3}
			\move{g}{4}
			\move{h}{4}
			\move{f}{6}
			\move{h}{6}
			\move{d}{6}
			\move{o}{4}
			\move{g}{3}
			\move{g}{2}
			\move{f}{2}
			\move{e}{2}
			\move{h}{2}
			\move{f}{1}
			\move{j}{2}
			\move{k}{3}
			\move{k}{2}
			\move{l}{3}
			\move{d}{2}
			\move{l}{2}
			\move{d}{1}
		\end{psgopartialboard}
		
		\vspace{1em}
		Blanc sacrifie le groupe \stone[3]{white} - \stone[7]{white} et renforce sa position sur le bord droit.
	}
	
\end{document}