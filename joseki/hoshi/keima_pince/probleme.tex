\documentclass[preview, border=0pt, varwidth=false]{standalone}

\usepackage[paperwidth=10.2cm, paperheight=14.65cm, margin=0cm]{geometry}
\usepackage{psgo}
\usepackage{unicode-math}
\setmathfont{XITSMath-Regular}

\begin{document}
	\setgounit{0.4cm} 
	
\parbox[c][14.65cm][c]{10.2cm}{
	\centering
	
	{\Large\textbf{Approche en keima pincée} $\medblackstar \medblackstar \medblackstar \medwhitestar$ \\ Blanc tenuki}
	\vspace{3em}
	
	\begin{psgopartialboard}{(1,1)(19,9)}
		\stone{white}{d}{4}
		\stone{black}{q}{4}
		\stone{black}{f}{3}
		\pass
		\move{o}{3}
		\move{m}{3}
		\move{c}{6}
	\end{psgopartialboard}
	
	\vspace{1em}
	Blanc tenuki parce qu'en l'état \stone[1]{white} n'est pas morte en un seul coup de noir.  \\

	\bigskip

	Quelle est la bonne extension pour noir sur le bord droit ? Il faut considérer deux réponses possibles différentes de blanc, en san-san ou en komoku.
}

\end{document}