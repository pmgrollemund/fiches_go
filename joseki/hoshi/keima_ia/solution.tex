\documentclass[preview, border=0pt, varwidth=false]{standalone}

\usepackage[paperwidth=10.2cm, paperheight=14.65cm, margin=0cm]{geometry}
\usepackage{psgo}

\begin{document}
	\setgounit{0.6cm} 
	
\parbox[c][14.65cm][c]{10.2cm}{
	\centering
	
	\begin{psgopartialboard}{(1,1)(9,7)}
		\stone{black}{d}{4}
		\stone{black}{d}{6}
		\stone{white}{f}{3}
		\markpos{\marktr}{c}{3}
		\pass
		\move{e}{5}
		\move{d}{5}
		\move{d}{2}
	\end{psgopartialboard}
	
	\vspace{1em}
	
	Le coup $\triangle$ est très gros, mais pas nécessairement urgent. 
	\vspace{1em}

	Si blanc omet le coup \stone[1]{white} et saute tout de suite en \stone[3]{white}, alors noir répond en $\triangle$, et le coup \stone[1]{white} ne sera plus forçant puisque $\triangle$ protège la coupe. 
}

\end{document}