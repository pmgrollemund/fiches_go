\documentclass[preview, border=0pt, varwidth=false]{standalone}

\usepackage[paperwidth=10.2cm, paperheight=14.65cm, margin=0cm]{geometry}
\usepackage{psgo}

\begin{document}
	\setgounit{0.6cm} 
	
\parbox[c][14.65cm][c]{10.2cm}{
	\centering
	
	%{\Large\textbf{Approche en Kema} \\ Blanc laisse le coin}
	%\vspace{3em}
	
	\begin{psgopartialboard}{(1,1)(11,7)}
		\stone{black}{d}{4}
		\stone{white}{f}{3}
		\stone{black}{h}{3}
		\pass
		\move{c}{3}
		\move{c}{4}
		\move{d}{3}
		\move{e}{4}
		\move{f}{2}
		\move{g}{5}
	\end{psgopartialboard}
	
	\vspace{1em}
	Blanc peut envahir le coin en jouant san-san \stone[1]{white} et \\ concède un mur noir après \stone[6]{black}. Le double keima  \\ autour de \stone[6]{black} contient des faiblesses à exploiter \\ plus tard dans la partie.
}

\end{document}