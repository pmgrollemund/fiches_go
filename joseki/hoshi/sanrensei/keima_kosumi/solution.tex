\documentclass[preview, border=0pt, varwidth=false]{standalone}

\usepackage[paperwidth=10.2cm, paperheight=14.65cm, margin=0cm]{geometry}
\usepackage{psgo}

\begin{document}
	\setgounit{0.6cm} 
	
\parbox[c][14.65cm][c]{10.2cm}{
	\centering
	
	\begin{psgopartialboard}{(1,1)(13,11)}
		\stone{black}{d}{4}
		\stone{black}{d}{10}
		\pass
		\move{f}{3}
		\move{e}{5}
		\move{c}{3}
		\move{d}{3}
		\move{c}{4}
		\move{c}{6}
		\move{b}{6}
		\move{b}{7}
		\move{b}{5}
		\move{c}{7}
		\move{d}{2}
		\move{e}{2}
		\move{c}{2}
	\end{psgopartialboard}
	
	\vspace{1em}
	Blanc prend le coin, noir récupère l'initiative et peut jouer un autre point important sur le plateau
}

\end{document}